\documentclass{article}

\title{Group Organisation Document}
\author{ESE Group E}

\begin{document}

\maketitle

\section{Team Roles}
We have decided on a role based model; Provisionally we have allocated team positions as follows:\\
\\
Daniel: Quality Assuror\\
Mohammed: Toolsmith\\
Shaun: Librarian\\
Simon: Manager\\\\
We have chosen the role based model over the others because we believe the ability for people to be in charge of different aspects of the project. By the time we come to write the team report, each person will know their role inside-out.

\section{authority}
\subsection{Who Decides?}
We have allocated a Manager role to Simon, so in this context ultimately the decisions will lie with the manager. However, we will discuss the pro's and con's of any decision as a team so that we can make an informed decision.

\subsection{How are Decisions taken?}
Ideas for the project can be brought up by any of members of the
team. These ideas would be itemized and discussed as
appropriately. Where there are conflicts regarding the final decision
to take then the team leader would have the final say. This is so that
we can save time and make advancement on the project.

\section{communication}

\subsection{Meeting Times}
As a start, we have chosen to meet in the Boyd Orr level 7 lab every Wedneday and Friday, from 9am until 11am. These will be fixed meetings and other times can be arranged to meet later if necessary.

\subsection{Ad-Hoc communication}
We all have each other's mobile phone numbers, and email addresses. We
can also  use instant messenger to communicate. The trac management
system allows us to communicate and track the progress of the project as well.

\section{Managing Information}
\subsection{Information Storage}
We have selected to have a dedicated subversion repository to manage
our source code revisions. We will be using the trac management system
to manage our project (ie. Allocate milestones, track bugs) and share
documents and specifications using the wiki.

\subsection{Distribution Channels}
Our subversion, and trac system can be accessed anywhere with an
internet connection. Therefore, our team will have access to the
sourcecode anywhere they have access to the internet. Changes can be
committed by any team member. At every modification of the
documentation or source code an email will be sent to the project
members to allow a easy way to keep them abreast of the progress.

\subsection{Data Access}
All members of the team will have access to the server and information. At the moment, anonymous access is not allowed. However, we will release an alpha version to the client - at which time they will be granted access to trac to allow them to submit bug reports. Any external beta testers we bring in, will also have access to the trac system, and read only access to our subversion repository.

\section{Risks to the Project}

\subsection{Communication Risks}
The communication risks can occur between members of the team (inter-personal) and between team members and the technology available.
The Inter-personal risks could be reduced to a bare minimum by ensuring strong correspondence by mail and by making calls on a frequent basis to inquire on the whereabouts of each team member.
Since we are using a dedicated server, there are risks involved in
that the server could crash or shutdown unexpectedly. However the
server has been configured to use a raid1 hard drive mirror to prevent
any hardware crash causing data loss. Frequent backups on an
external server will be used to reduce the hardware problems to a minimum.

\subsection{Team Organisation Risks}
Because we have chosen a role based organisational model there is the risk that someone may fall out of contact (due to unforseen circumstances). To counter this, we have decided that all changes we make to the source code and all documents we edit will be stored on the server. This means that we dont run the risk of one person disappearing with all the information that we need. It does not however negate the problem that if someone disappears, then someone else will have to do the work for it.
\end{document}
