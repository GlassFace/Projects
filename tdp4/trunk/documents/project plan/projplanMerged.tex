% This is a template for constructing your project plan document, but
% also to show the use of the l3deliverable class. Use pdflatex and
% bibtex to process the file, creating a PDF file as output (there is
% no need to use dvips when using pdflatex).
%
% Several meta data commands have been implemented to collect
% information such as deliverable identifier, project name etc (see
% below the \date command.

%todo proof read, add picture of how it should look
%todo rearrange glossary
\documentclass{article}

\usepackage{a4wide}
\usepackage{graphicx}
\usepackage{bigstrut}


% You can use this package to automatically insert version control
% information into your document (an example of how to do this is
% shown below).  Make sure to set the 'svn:keywords' subversion
% property to 'Id' for the source file, for example:
%
% svn propset svn:keywords 'Id' deliverable-2.tex
%
% The information between the two $$ will now be updated when you nexthttp://fims.moodle.gla.ac.uk/mod/resource/view.php?id=7005
% commit the file to your SVN repository.
%
% You can of course, just use this field to insert manual version
% information, e.g. 1.2, 1.2.1 ... instead.
%\usepackage{svn-multi}
%\svnid{$Id: deliverable-3-template.tex 1297 2009-11-06 21:39:10Z tws $}
%\version{SVN Revision \svnrev~ \
%
%Made \svnday/\svnmonth/\svnyear~ by \svnauthor}

\usepackage{tabularx}
\usepackage{url}

%todo change version after submission

\title{Project Plan and Risk Management}

\author{Esiri Igbako
        Amal Kakaiya
        Simon Jouet\\
        }

\date{20 January 2011}

\newpage

\begin{document}

\maketitle

\tableofcontents

\newpage

\section{Team Organisation}
\subsection{Introduction}
The team name is ‘Team LEAD’ where LEAD stands for “Linux Embedded Automotive
Dashboard”. The team consists of three members:
\begin{itemize}
	\item Amal Kakaiya
	\item Simon Jouet
	\item Esiri Igbako
\end{itemize} 

An original fourth member, left the team and course due to other project commitments.

\subsection{Communication}
Weekly meetings of the team, and meeting with project supervisors, will track
the project progress against the team Gantt chart. These meetings will be used 
to assess deadlines, revise requirements, delegate tasks, and organise meetings 
with UGRacing (the client) should any clarification of requirements be needed.

\subsection{Development and Tracking}
Outside of formal meeting times, team members are expected to perform delegated 
tasks to deadlines and track progress through a ticketing system, Trac. This, 
along with the Subversion (SVN) versioning repository will ensure that the team 
has a well documented and transparent development process. Weekly time-sheets 
will also be filled in to match Trac tickets for further documentation. All 
team work, including code, documents, schematics, diagrams, data-sheets and 
sources are required to be added to the SVN repository throughout the 
development process. Not only is this a safe and secure means of keeping the 
data (providing the repository is backed up), it also allows all team members 
to work on up-to-date versions of all documents in the development process. 
Which allows concurrent development on code and documents.

\section{Risk Management Plan}
As part of the development process of the proposed system, the team has 
calculated the following potential risks, and probabilities for the risks 
becoming real, along with the forecasted consequences.

\subsection{Risk 1: Component Compatibility Issues}
Concerns have been raised about the fact that even though the components we 
have selected may conform to the same standards (for CAN), they might not work 
together as expected. Namely the Microchip and Freescale components.\\
\\
Probability: Medium - High\\
Impact: Medium\\
Contingency actions: Select different components, and work on 
finding/deveoping Linux drivers for them.

\subsection{Risk 2: Component failure}
This is mostly concerning the main COM Gumstix component. If this fails, then 
the entire project is compromised as the device is expensive and takes a while 
to deliver.\\
\\
Probability: Medium\\
Impact: High\\
Contingency actions: Multiple Gumstix (2 max) could be ordered just in case. 
However, this is expensive, so it will not be done. Other components can easily 
be ordered as they are fairly inexpensive and quick to obtain. 

\subsection{Risk 3: Team integration}
Poor inter-team communication (with teams developing other components) means 
that we would be may be to interpret data generated by other teams. Since we 
are collecting data from all teams, for display on our dashboard, standards 
must be agreed and adhered to in order for the project to succeed.\\
\\
Probability: Medium\\
Impact: High\\
Contingency actions: Agree standards with other teams and adapt code accordingly.

\subsection{Risk 4: Personnel loss}
In the event that a team member leaves the team and/or course, we must be able 
to prepare for the remaining workload.\\
\\
Probability: Low\\
Impact: High\\
Contingency actions: Re-distribution of workload. Potential case for extension 
of deadlines. Revised Gantt chart required.

\subsection{Risk 5: Misunderstood requirements}
Our understanding of UGRacings requirements for the dashboard, may be different 
from their understanding. \\
\\
Probability: Medium\\
Impact: Varying\\
To ensure that both sides have equal understanding, an iterative requirements 
gathering process can be used until an agreement is met.

\section{Conclusion}
The team has put a framework in place to allow for smooth development and 
delivery to the client. Risks have appropriate contingency plans, and the team 
should be able to deal with these forecasted risks should they become real. 
Unexpected risks will require further risk assessment and plan revision, 
depending on the severity of their impact.


\section{Project Plan}

\subsubsection{Identification}

This is the Project Plan document for the Linux Embedded Automotive Dashboard (LEAD) team. It contains the plan and the schedules for different tasks and who would be the leader of each tasks. It is susceptible to change as the project requirements change or unforeseen circumstances occur.
 
\subsubsection{The Team}

The Team that was assigned this Project consists of:
\begin{itemize}
\item Simon Jouet
\item Amal Kakaiya
\item Esiri Igbako
\end{itemize}
Dingkun Ren has defected from the team due to other responsibilities.


\subsection {Team Responsibilities}

We decided to assign ourselves some preliminary roles within the team. These roles are to ensure that the team is coordinating and project work runs as smoothly as possible. The roles within the team are stated below:

\subsubsection {Manager}
The role of manager was assigned to Amal Kakaiya. As the manager his task is to coordinate the project and make sure it is moving at a pace that would lead to its timely completion. He is also responsible for liaising with the client on behalf the team. This could be for requirements gathering or for validation of the teams work.
Authority

\subsubsection {Quality Assuror}
The role of quality assuror was assigned to Esiri Igbako. The Quality assuror is in charge of testing that our product meets requirements and that bugs in software are kept minimal. He will be in charge during our testing phase of development. He would also be responsible for editing and proof-reading documents before they are submitted.

\subsubsection {Librarian}
The role of librarian was assigned to Simon Jouet. The Librarian is in charge of ensuring that all documentation is kept organised. Also he would be responsible for managing and keeping track of the changes that are made in the documentation and codebase.

\subsection {Authority}

The team has allocated a Manager role to Amal, so in this context ultimately the decisions will lie with the manager. However, we will deliberate on any matters that arise as a team so that an informed decision can be made. 

Where there are conflicts regarding the final decision to take then the
Manager would have the final say. This may introduce a problem whereby some members of the team feel unsatisfied with a final decision. However, we believe we will be able to discuss any conflicts constructively and come to an agreement without this being necessary.

\subsection {Communication}

\subsubsection {Meeting Times}

As a start, we have chosen to meet in the project laboratory in the Rankine building every Monday and Wednesday from 2:00pm to 4:45pm. These will be fixed meetings times. As the project begins to gather momentum then the team would have to schedule more time to work on the project. This would be discussed when necessary.

\subsubsection {Ad-Hoc communication}

The team members have all taken note of each other’s contact details, mobile phone numbers, and email addresses. We would also frequently us instant messaging to communicate. Also, the trac management system allows us to communicate and track the progress of the project as well.

\subsubsection {Communication with client}
Interactions with the client is carried out through our allocated Moodle forum. However we are presently unable to post to the forum. This problem has been raised on several occasions to the project coordinator but we are still unable to post to the forum. The team’s alternate solution at the moment is use the email of the client or to try and organise meeting with the client.


\subsection {Documentation and Software Change Management Plan}
\subsection {Information Management}

The team decided to have a dedicated subversion repository hosted on Simon’s personal server to manage our source code revisions. We will be using the trac management system to manage our project. This trac management system would help to allocate milestones and track bugs. The team would also be using the trac Wiki to share documents and specifications.

\subsubsection {Information Access}

Our subversion, and trac system can be accessed anywhere with an internet connection. Therefore, our team will have access to the codebase anywhere they have access to the internet. Changes can be committed by any team member. At every modification of the documentation or source code an email will be sent to the project members to keep them informed with the progress.

\subsection {Task Allocation}
Tasks have been broken down to ensure that every member of the team is kept reasonably busy throughout the duration of the project. The tasks were assigned according to preference of each individual in the team.

Predevelopment Tasks

\begin{itemize}
\item Meeting with client to discuss specification
\item Component Selection
\item Research on components
\item Procurement of components
\end{itemize}

The team has just concluded the requirements gathering stage. However further interaction with the client may be needed to verify the validity of the requirements set. The team has also concluded the component selection phase and have made efforts to place orders for the desired components.
We would need to wait until the required components have been delivered before hardware fabrication and testing can be done. In the meantime, the software would be developed and tested to see that it fulfils the requirements of our client.
	
\subsection {Work Breakdown Structure}

Hardware Tasks:

\begin{itemize}
\item Circuit design
\item PCB design
\item Module Testing
\item Power Supply
\item Component soldering and Hardware fabrication
\item Hardware testing 
\item Weather-proof packaging design
\end{itemize}

Software Tasks:
\begin {itemize}
\item Drivers development
\begin {itemize}
\item Socket driver for CAN
\item Keyboard driver using pushbuttons
\end {itemize}
\item HSC08 code development
\item GUI Prototyping
\item X11 GUI development
\item Interface implementation between GUI and driver
\end {itemize}

\subsection {PERT Chart Illustration}


\begin{tabular}{|l|l|l|l|l}
\hline
Task & Symbol & Days & Overseer \bigstrut \\ \hline
Circuit design	& T1	& 7	& Simon \bigstrut \\ \hline
GUI Prototyping	& T2	& 1 & Esiri \bigstrut[t] \\
X11 GUI dev	& T3 	& 7	& Esiri \bigstrut[b] \\ \hline
PCB design	& T4	& 4	& Amal \bigstrut[b] \\ \hline
Module Testing	& T5 & 2 & Simon \bigstrut[b] \\ \hline
Power Supply	& T6 & 2 & Amal \bigstrut[b] \\ \hline
Component Soldering	& T7 & 2	& Amal \bigstrut[b] \\ \hline
Fabrication Of PCB	& T8	& 3	& Simon \bigstrut[b] \\ \hline
Socket driver dev	& T9	& 3	& Amal \bigstrut[b] \\ \hline
Keyboard driver dev	& T10	& 4	& Esiri \bigstrut[b] \\ \hline
GUI and driver interface impl	& T11	& 6	& Esiri \bigstrut[b] \\ \hline
Weather-proof packaging design	& T12	& 2 & Amal \bigstrut[b] \\ \hline
\end{tabular}

NB: A working day has been realistically estimated to be 5 hours.

Although testing has not been mentioned as a task, extensive testing would be carried out at the end of each of task. The time resource allocated to a task would invariably include time for testing when the task is completed. The logic behind splitting the tasks to software and hardware is to allow both to run simultaneously, thus ending in a uniformed fashion.
\end{document}
